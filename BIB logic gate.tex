\documentclass{article}
\usepackage{graphicx}

\begin{document}
	\title{Logic Gates}
	\author{Kamara Nsofor}

	
	
\maketitle
\newpage
\section{Table Of Contents}
	
	\begin{itemize}
		\item Logic Gate
		\item Type of Logic Gates
		\item Truth Tables
		\item Summary		
	\end{itemize}
	
	
\newpage
\section{Logic Gate}
	A logic gate is a building block of a digital circuit which is at the heart of any computer. Behind every digital system is a logic gate.\\
	Logic gates perform logical operartions that take binary input (0s and 1s) and produce a single binary output. They are used in most electronic device like smartphone, Tablets Memeory devices~\cite{}.
	
\section{Type of Logic Gates}
		The type of logic gate are:
		\begin{itemize}
			\item And Gate
			\item OR Gate
			\item NOT Gate
			\item NAND Gate
			\item NOR Gate
			\item XOR Gate
			\item XNOR Gate
		\end{itemize}
	
\subsection{AND Gate}
\includegraphics[width= 0.3\linewidth]{And Gate}%picture of And gate
The AND gate produces a true output (result of 1) only for the single case when all of the input variables are 1 and a false output(result of 0) where one or more inputs are 0.
\subsection{OR Gate}
\includegraphics[width= 0.3\linewidth]{OR-Gate}%picture of OR gate
The OR gate produces a true output (result of 1) when any of the input variable is 1 and a false output (result of 0)only when all the input variable are 0.
\subsection{NOT Gate}
\includegraphics[width= 0.3\linewidth]{Not Gate}%picture of NOT gate
The NOT gate reverses the original input to give an inverted output.
\subsection{NOR Gate}
\includegraphics[width= 0.3\linewidth]{NOR-Gate}%picture of NOR gate
The NOR gate gives a true output (result of 1)only when both inputs are false(0).
\subsection{NAND Gate}
\includegraphics[width= 0.3\linewidth]{NAND-Gate}%picture of NAND gate
The NAND gate gives a false output (result of 0)when both inputs are true(1).
\subsection{XOR Gate}
\includegraphics[width= 0.7\linewidth]{XOR-Gate}%picture of XOR gate
The XOR gate Gives a true output (result of 1) if one, and only one, of the inputs to the gate is true (1).
\subsection{XNOR Gate}
\includegraphics[width= 0.3\linewidth]{XNOR-Gate}%picture of XNOR gate
The XNOR gate gives a true output (1), if the inputs are the same, and a false output (0) if the inputs are different.
		
		
		
\section{Tables}
\begin{table}
	\begin{center}
		\label{tab:and}
		\caption{AND-Table}
		\begin{tabular}{|c|c|c|}
			A & B & A AND B\\
			\hline
			1&1&1\\
			1&0&0\\
			0&1&0\\
			0&0&0\\
		\end{tabular}
	\end{center}
\end{table}

\begin{table}
	\begin{center}
		\label{tab:or}
		\caption{OR-Table}
		\begin{tabular}{|c|c|c|}
			A & B & A OR B\\
			\hline
			1&1&1\\
			1&0&1\\
			0&1&1\\
			0&0&0\\
		\end{tabular}
	\end{center}
\end{table}

\begin{table}
	\begin{center}
		\label{tab:not}
		\caption{NOT-Table}
		\begin{tabular}{|c|c|}
			A & NOT A\\
			\hline
			1&0\\
			0&1\\
		\end{tabular}
	\end{center}
\end{table}

\begin{table}
	\begin{center}
		\label{tab:table4}
		\caption{NAND-Table}
		\begin{tabular}{|c|c|c|}
			A& B & A NAND B\\
			\hline
			1&1&0\\
			1&0&1\\
			0&1&1\\
			0&0&1\\
		\end{tabular}
	\end{center}
\end{table}

\begin{table}
	\begin{center}
		\label{tab:table5}
		\caption{NOR-Table}
		\begin{tabular}{|c|c|c|}
			A & B & A NOR B\\
			\hline
			1&1&0\\
			1&0&0\\
			0&1&0\\
			0&0&1\\
		\end{tabular}
	\end{center}
\end{table}

\begin{table}
	\begin{center}
		\label{tab:table6}
		\caption{XOR-Table}
		\begin{tabular}{|c|c|c|}
			A & B &A XOR B\\
			\hline
			1&1&0\\
			1&0&1\\
			0&1&1\\
			0&0&0\\
		\end{tabular}
	\end{center}
\end{table}

\begin{table}
	\begin{center}
		\label{tab:table7}
		\caption{XNOR-Table}
		\begin{tabular}{|c|c|c|}
			A & B & A XNOR B\\
			\hline
			1&1&1\\
			1&0&0\\
			0&1&0\\
			0&0&1\\
		\end{tabular}
	\end{center}
\end{table}
\newpage
\section{Summary}
%summary
referenced from\cite{Okoacha2021logig}

\bibliography{bib}
\bibliographystyle{ieeetr}

\end{document}